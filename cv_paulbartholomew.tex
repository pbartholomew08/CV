\documentclass[10pt,a4paper]{moderncv} % Font sizes: 10, 11, or 12; paper sizes: a4paper, letterpaper, a5paper, legalpaper, executivepaper or landscape; font families: sans or roman

\moderncvstyle{classic} % CV theme - options include: 'casual' (default), 'classic', 'oldstyle' and 'banking'
\moderncvcolor{black} % CV color - options include: 'blue' (default), 'orange', 'green', 'red', 'purple', 'grey' and 'black'

\usepackage[utf8]{inputenc}
\usepackage{geometry} % Reduce document margins
\geometry{a4paper,tmargin=1.2cm,bmargin=1.5cm,lmargin=1.5cm,rmargin=1.5cm,headheight=2.2cm,headsep=0.5cm,footskip=0.5cm}
\setlength{\hintscolumnwidth}{1.8cm} % Uncomment to change the width of the dates column
%\setlength{\makecvtitlenamewidth}{10cm} % For the 'classic' style, uncomment to adjust the width of the space allocated to your name
\renewcommand\familydefault{cmss}

\firstname{Paul} % Your first name
\familyname{Bartholomew} % Your last name

\address{57 Antonine Road}{Bearsden, Glasgow, G61 4DS}
\mobile{+44 (0)7722926927}
\email{p.bartholomew@epcc.ed.ac.uk}

%----------------------------------------------------------------------------------------

\begin{document}

\makecvtitle % Print the CV title
\vspace{-0.6cm}

%\section{Application For}
%\cvitem{}{{\fontsize{12}{12.5}\selectfont Application for eCSE funding}}
%\vspace{5mm}

I am an applications consultant in HPC at EPCC, University of Edinburgh with over six years' experience
working with, and developing, Computational Fluid Dynamics (CFD) software for research and industry.
In the course of this research I have developed extensive skills working with research software and
programming in widely used scientific languages including \texttt{C}, \texttt{Fortran} and
\texttt{Python} and a passion for computational science.

%----------------------------------------------------------------------------------------
%	EDUCATION SECTION
%----------------------------------------------------------------------------------------

\section{Education}

\cventry{2013 -- 2017}{PhD in Computational Fluid Dynamics}{Department of Mechanical Engineering,
  Imperial College London}{Title: Development of an implicit framework for the two-fluid model on
  unstructured grids}{}{
  \begin{itemize}
  \item Contributed to the development of a general-purpose CFD research code, implementing a
    fully-coupled solver for the two-fluid model.
    \begin{itemize}
    \item Learned to write \texttt{C} and write software intended for parallel computing
    \end{itemize}
  \item Published in international peer-reviewed journal
  \item Presented work at national and international conferences
  \item Contributed to teaching of MSc students as a Graduate Teaching Assistant for the CFD course
  \end{itemize}
}
\cventry{2009 -- 2013}{MEng. (First class) in Mechanical Engineering}{Imperial College London}{}{}{
  \begin{itemize}
  \item Achieved the 3rd year Dean's list
  \end{itemize}
}
\cventry{2007 -- 2008}{Advanced Highers}{AAA in Chemistry, Maths and Physics}{}{}{
  \begin{itemize}
  \item Senior Proxime Accessit, Bearsden Academy
  \end{itemize}
}
\cventry{2006 -- 2007}{Highers}{AAAA in Chemistry, Maths, Physics and Technical Studies; B in
  English}{}{}{}
\cventry{2004 -- 2006}{Standard Grades}{Grade 1 in Art, Chemistry, French, Geography, Maths, Physics
  and Technical Studies; grade 2 in English}{}{}{}

\section{Professional/research experience}

\cventry{2019 -- pres.}{Applications consultant in HPC}{}{EPCC, University of Edinurgh}{}{
  \begin{itemize}
  \item Developed good working relationships working with external partners, both commercial and
    academic
  \item Gained experience profiling and analysing codes to diagnose performance issues for HPC
  \item Proposed and supervised MSc projects
  \end{itemize}
}
\cventry{2018 -- 2019}{eCSE 13-03 \textit{A high-order accurate solver for free-surface
    flows}}{}{Imperial College London}{}{
  \begin{itemize}
  \item Awarded additional funding to continue work of eCSE 10-02 to develop a free-surface solver
  \item Key contributor to project to modernise \texttt{Incompact3d} codebase
  \item Presented work at national and international conferences
  \end{itemize}
}
\cventry{2017 -- 2018}{eCSE 10-02 \textit{An adjoint solver for variable-density flows in the low Mach
    number limit}}{}{Imperial College London}{}{
  \begin{itemize}
  \item Implemented a low Mach number solver in open-source CFD code \texttt{Incompact3d}
    \begin{itemize}
    \item Learned to write \texttt{Fortran}
    \item Gained experience working with Tier-1/0 super computers
    \end{itemize}
    \item Presented work at national conferences
    \item Published in international peer-reviewed journal
  \end{itemize}
}

\cventry{2012}{Undergraduate Research Opportunities Programme}{}{Imperial College London}{}{
  \begin{itemize}
  \item Won funding to join a research group in the Thermofluids division of the Mechanical
    Engineering department at Imperial College for the summer between the third and final year of my
    undergraduate MEng. degree.
  \item Gained experience working with Paraview
  \item Developed simple CFD code in Python
  \end{itemize}
}

\cventry{2008 -- 2009}{Year In Industry}{BAE Systems}{Glasgow}{}{
  \begin{itemize}
  \item Worked in the operations department at the Scotstoun shipyard
  \item Implemented a requisition tracking system to facilitate transfer of materials between
    projects
  \item Assisted in project management of the charity project to refit the Seagull barge
  \end{itemize}
}

\section{Publications}

Include:

\cventry{}{P. Bartholomew, G. Deskos, R. Frantz, F. Schuch, E. Lamballais, S. Laizet}{Xcompact3d: An
  open-source framework for solving turbulence problems on a Cartesian mesh in SoftwareX
}{2020}{}{}
\cventry{}{P. Bartholomew, S. Laizet}{A New Highly Scalable, High-Order Accurate Framework for
  Variable-Density Flows: Application to Non-Boussinesq Gravity Currents in Computer Physics
  Communications
}{2019}{}{}
\cventry{}{P. Bartholomew, F. Denner, M. H. Abdol-Azis, A. Marquis, B. van Wachem}{Unified
  Formulation of the Momentum-Weighted Interpolation for Collocated Variable Arrangements in Journal
  of Computational Physics
}{2018}{}{}

% \pagebreak

\section{Skills and interests}

\cventry{}{Technical knowledge}{}{}{}{
  \begin{itemize}
  \item Very strong background in numerical software, particularly CFD, having worked with
    and extended two research codes each using different numerical methods
  \item Contributed to the development of new software
  \item Able to effectively troubleshoot problems in simulations\\
  \end{itemize}
}
\cventry{}{Computer programming and skills}{}{}{}{
  \begin{itemize}
  \item Have developed software in several of the major languages used in computational science
    including \texttt{C}, \texttt{Fortran} and \texttt{Python} and have experience using
    \texttt{MATLAB}/\texttt{Octave}
  \item Have experience programming for distributed systems and a good knowledge of \texttt{MPI} and
    the \texttt{PETSc} library
  \item Familiar with Linux use and administration
  \item Experience using tier-1/0 HPC systems\\
  \end{itemize}
}
\cventry{}{Communication skills}{}{}{}{
  \begin{itemize}
  \item Strong presentation skills developed by presenting to both specialist and non-specialist
    audiences at national and international conferences
  \item Ability to explain concepts clearly, honed by working as a Graduate Teaching Assistant
    during my PhD
  \item Contributed to the development of team members by producing a short introduction to git to
    present to PhD students in the group
  \item Precise and clear writing skills developed by publishing work in peer-reviewed journals
  \item Combining these skills with software development has great potential for scientific software
    \begin{itemize}
    \item I have applied this with literate programming techniques to produce reports with
      integrated post-processing code, improving the presentation of the analysis and enabling
      reproducibility of the report \\
    \end{itemize}
  \end{itemize}
}
\cventry{}{Teamwork and collaboration}{}{}{}{
  \begin{itemize}
  \item Good teamwork skills having worked as part of a team during both my PhD and PDRA positions
    working on common software for research
  \item Experience working with source control management tools including git and svn
  \item Identification of areas for improvement
    \begin{itemize}
    \item Instigated development of a post-processing library in collaboration with another PDRA,
      based on Python tools we had developed to encourage shared approaches to data analysis.
      This has contributed to the work of other team members \\
    \end{itemize}
  \end{itemize}
}
\cventry{}{Personal interests}{}{}{}{
  \begin{itemize}
  \item \textbf{Sports} Cycling, badminton and squash
  \item \textbf{Hobbies} Guitar\\
  \end{itemize}
}

% \section{Conference and seminar presentations}

% \cventry{May 2019}{P. Bartholomew, G. Deskos, S. Laizet}{Xcompact3d: A Powerful Framework to Study
%   Turbulent Flows with Turbulence-Resolving Simulations}{EuroHPC, 13-17 May 2019}{Pozna\'{n},
%   Poland}{}
% \cventry{Apr 2019}{P. Bartholomew, S. Laizet}{Modernising and expanding the capabilities of the
%   high-order flow solver Incompact3d (Poster)}{Numerical Algorithms for High-Performance
%   Computational Science, 8-9 February 2019}{London, United-Kingdom}{}
% \cventry{Feb 2019}{P. Bartholomew, S. Laizet}{Modernising and expanding the capabilities of the
%   high-order flow solver Incompact3d (Poster)}{RSLondonSouthEast Workshop 2019, 7 February
%   2019}{London, United-Kingdom}{}
% \cventry{Sep 2018}{P. Bartholomew, S. Laizet}{QuasIncompact3D: A Highly Scaleable Solver for Navier
%   Stokes in the Low Mach Number Limit}{UK Fluids Conference, 4-6 September 2018}{Manchester,
%   United-Kingdom}{}
% \cventry{Apr 2018}{P. Bartholomew, S. Laizet}{Simulations of Variable-Density Flows in the Low Mach
%   Number Limit}{Incompact3D User Group Meeting 2018, April
%   2018}{London, United-Kingdom}{}
% \cventry{Sep 2016}{P. Bartholomew, A. Marquis, B. van Wachem}{Modelling of
%   Turbulent Gas-Solid Flows in the Eulerian Framework}{Inaugural UK Fluids
%   Conference 2016, 7-9 September 2016}{London, United-Kingdom}{}

% \cventry{May 2016}{P. Bartholomew, A. Marquis, F. Denner, B. van Wachem}{Development of a Fully
%   Coupled Solver for the Eulerian-Eulerian Approach}{9th International Conference on Multiphase Flow
%   (ICMF 2016), 22-27 May 2016}{Florence, Italy}{}

% \section{Teaching activities}

% \cventry{2015-2017}{CDT - Computational Fluid Dynamics}{Teaching assistant}{Imperial College London}{}{}

% \vspace{0.3cm}
% \begin{minipage}[t]{0.45\textwidth}
% \section{Computer skills}
% \cvitemwithcomment{Languages}{C, Fortran, Python, \LaTeX, Matlab}{}
% \cvitemwithcomment{Software}{Linux, MPI, Paraview, PETSc}{}
% \end{minipage}
% \hspace{0.7cm}
% \begin{minipage}[t]{0.45\textwidth}
% \setlength{\hintscolumnwidth}{3.4cm} 
% \setlength{\hintscolumnwidth}{3cm} 
% \end{minipage}

% %----------------------------------------------------------------------------------------
% %	INTERESTS SECTION
% % ----------------------------------------------------------------------------------------

% \section{Interests}

% \renewcommand{\listitemsymbol}{$\bullet$~} % Changes the symbol used for lists

% % \cvitemwithcomment{Sports}{Running, Coaching, Rowing}{}
% \cvitemwithcomment{Sports}{Cycling, Badminton, Squash}{}
% \cvitemwithcomment{Hobbies}{Guitar}{}

\section{Referees}

Available on demand.

% \cventry{}{NAME}{POSITION}{\newline{}DEPARTMENT,\newline{}ADDRESS,\newline{}TEL
% NO.\newline{}EMAIL}{}{}

\end{document}

%%% Local Variables:
%%% mode: latex
%%% TeX-master: t
%%% End:
