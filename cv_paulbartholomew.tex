\documentclass[10pt,a4paper]{moderncv} % Font sizes: 10, 11, or 12; paper sizes: a4paper, letterpaper, a5paper, legalpaper, executivepaper or landscape; font families: sans or roman

\moderncvstyle{classic} % CV theme - options include: 'casual' (default), 'classic', 'oldstyle' and 'banking'
\moderncvcolor{black} % CV color - options include: 'blue' (default), 'orange', 'green', 'red', 'purple', 'grey' and 'black'

\usepackage[utf8]{inputenc}
\usepackage{geometry} % Reduce document margins
\geometry{a4paper,tmargin=1.2cm,bmargin=1.5cm,lmargin=1.5cm,rmargin=1.5cm,headheight=2.2cm,headsep=0.5cm,footskip=0.5cm}
%\setlength{\hintscolumnwidth}{3cm} % Uncomment to change the width of the dates column
%\setlength{\makecvtitlenamewidth}{10cm} % For the 'classic' style, uncomment to adjust the width of the space allocated to your name
\renewcommand\familydefault{cmss}

\firstname{Paul} % Your first name
\familyname{Bartholomew} % Your last name

\address{26 Viewfield Road}{London, SW18 5JE}
\mobile{+44 (0)7722926927}
\email{paul.bartholomew08@imperial.ac.uk}

%----------------------------------------------------------------------------------------

\begin{document}

\makecvtitle % Print the CV title
\vspace{-0.6cm}

%\section{Application For}
%\cvitem{}{{\fontsize{12}{12.5}\selectfont Application for eCSE funding}}
%\vspace{5mm}

I am a research assistant at Imperial College London, currently working on the eCSE project `An
adjoint solver for variable-density flows in the low Mach number limit' and previously completed a
PhD at Imperial College London, working on the development of a fully-coupled solver for the
two-fluid model of multiphase flows.
In the course of my research, I have implemented a variable-density solver for Incompact3D and
pressure-velocity implicit and pressure-velocity-volume fraction implicit solvers for the two-fluid
model on collocated, unstructured meshes in MultiFlow.
Additionally, my research interests have covered Momentum Weighted Interpolation and its application
to multiphase flows, particularly the theoretical basis for, and justification of, its use.
I have a broad experience of working with parallel, general purpose CFD codes developed in-house
and have contributed to their development in a range of areas including: IO, automated meshing and
investigating preconditioners in addition to the implementation of the aforementioned solvers.
% In addition to my research work I am also responsible for the installation and maintenance of the
% van Wachem research group's computers.
\\

%----------------------------------------------------------------------------------------
%	EDUCATION SECTION
%----------------------------------------------------------------------------------------

\section{Education}

\cventry{2013 -- 2017}{PhD in Computational Fluid Dynamics}{Imperial College London}{}{}{
	Supervisors: B. van Wachem (Imperial College), A. Marquis (Imperial College).}
\cventry{2009 -- 2013}{MEng. (First class) in Mechanical
Engineering}{Imperial College London}{}{}{}  % Arguments not required can be left empty
\cventry{2007 -- 2008}{Advanced Highers}{AAA in Chemistry, Maths and Physics}{}{}{}
\cventry{2006 -- 2007}{Highers}{AAAA in Chemistry, Maths, Physics and Technical Studies; B in English}{}{}{}
\cventry{2004 -- 2006}{Standard Grades}{Grade 1 in Art, Chemistry, French,
Geography, Maths, Physics and Technical Studies; grade 2 in English}{}{}{}

\cventry{}{Academic Prizes}{\newline{}3rd year Dean's list, Mechanical
Engineering\newline{}Senior Proxime Accessit, Bearsden Academy}{}{}{}

\section{Professional/research experience}

\cventry{2018 -- 2019}{eCSE 13-03 \textit{A something something}}{}{Imperial College London}{}{
  Supervisor: S. Laizet, Imperial College London\newline{}
  Since October 2018 I have been working on
}
\cventry{2017 -- 2018}{eCSE 10-02 \textit{An adjoint solver for variable-density flows in the low Mach
number limit}}{}{Imperial College London}{}{
Supervisor: S. Laizet, Imperial College London\newline{}
This project was focused on the implementation of a Low Mach Number solver in the open-source CFD
code Incompact3D, available at: \url{https://github.com/ptb0890/Quasincompact3d}.
In the course of this project I have gained experience programming in Fortran and using
Tier-0/Tier-1 super computers.
I have presented intermediate results from the project at the Incompact3D User Group 2018 meeting at
Imperial college, and will present further results at the UK Fluids Conference 2018 at the
University of Manchester.
This work has lead to the publication of a peer-reviewed journal paper (\textit{Bartholomew \&
  Laizet 2019}).
}

\cventry{2012}{Undergraduate Research Opportunities Programme}{}{Imperial College London}{}{
Supervisor: A. Marquis, Imperial College London\newline{}
I obtained funding to join a research group in the Thermofluids division of the
Mechanical Engineering department at Imperial College for the summer between the
third and final year of my undergraduate MEng. degree.
During my project I looked at automating analysis with Paraview using the Python API
and developed a CFD code based on the SIMPLE algorithm.
}

\cventry{2008 -- 2009}{Year In Industry}{BAE Systems}{Glasgow}{}{
Supervisor: A. McNally \newline{}
Between finishing high school and starting university I spent a year working in
the operations department at the BAE Systems shipyard in Scotstoun, Glasgow.
During my time there I implemented a requisition tracking system to facilitate
the transfer of materials between projects and assisted in project management
of the Seagull barge refit for charity.}

\section{Publications}

\cventry{}{P. Bartholomew, S. Laizet}{A New Highly Scalable, High-Order Accurate Framework for
  Variable-Density Flows: Application to Non-Boussinesq Gravity currents in Computer Physics
  Communications}{2019}{}{}
\cventry{}{P. Bartholomew, F. Denner, M. H. Abdol-Azis, A. Marquis, B. van Wachem}{Unified
  Formulation of the Momentum-Weighted Interpolation for Collocated Variable Arrangements in Journal
  of Computational Physics}{2018}{}{}

\pagebreak

\section{Conference and seminar presentations}

\cventry{May 2019}{P. Bartholomew, G. Deskos, S. Laizet}{Xcompact3d: A Powerful Framework to Study
  Turbulent Flows with Turbulence-Resolving Simulations}{EuroHPC, 13-17 May 2019}{Pozna\'{n},
  Poland}{}
\cventry{Sep 2018}{P. Bartholomew, S. Laizet}{QuasIncompact3D: A Highly Scaleable Solver for Navier
  Stokes in the Low Mach Number Limit}{UK Fluids Conference, 4-6 September 2018}{Manchester,
  United-Kingdom}{}
\cventry{Apr 2018}{P. Bartholomew, S. Laizet}{Simulations of Variable-Density Flows in the Low Mach
  Number Limit}{Incompact3D User Group Meeting 2018, April
  2018}{London, United-Kingdom}{}
\cventry{Sep 2016}{P. Bartholomew, A. Marquis, B. van Wachem}{Modelling of
  Turbulent Gas-Solid Flows in the Eulerian Framework}{Inaugural UK Fluids
  Conference 2016, 7-9 September 2016}{London, United-Kingdom}{}

\cventry{May 2016}{P. Bartholomew, A. Marquis, F. Denner, B. van
Wachem}{Development of a Fully Coupled Solver for the Eulerian-Eulerian
Approach}{9th International Conference on Multiphase Flow (ICMF 2016), 22-27
May 2016}{Florence, Italy}{}

\section{Teaching activities}

\cventry{2015-2017}{CDT - Computational Fluid Dynamics}{Teaching assistant}{Imperial College London}{}{}

\vspace{0.3cm}
\begin{minipage}[t]{0.45\textwidth}
\section{Computer skills}
\cvitemwithcomment{Languages}{C, Fortran, MPI, Python, \LaTeX}{}
\cvitemwithcomment{Software}{Linux, Matlab, Paraview, PETSc}{}
\end{minipage}
\hspace{0.7cm}
\begin{minipage}[t]{0.45\textwidth}
\setlength{\hintscolumnwidth}{3.4cm} 
\setlength{\hintscolumnwidth}{3cm} 
\end{minipage}

%----------------------------------------------------------------------------------------
%	INTERESTS SECTION
%----------------------------------------------------------------------------------------

\section{Interests}

\renewcommand{\listitemsymbol}{$\bullet$~} % Changes the symbol used for lists

% \cvitemwithcomment{Sports}{Running, Coaching, Rowing}{}
\cvitemwithcomment{Sports}{Cycling, Swimming}{}
\cvitemwithcomment{Hobbies}{Guitar}{}

\section{Referees}

% \cventry{}{Prof. Berend van Wachem}{Professor of Multiphase Flow
% Modelling}{\newline{}Dept. Mechanical Engineering,\newline{}622,
% City and Guilds Building, South Kensington Campus, Imperial College London\newline{}020 7594
% 7030\newline{}b.van-wachem@imperial.ac.uk}{}{}
\cventry{}{Dr. Sylvain Laizet}{Senior Lecturer}{\newline{}Dept. Aeronautical
Engineering,\newline{}339, City
and Guilds Building, South Kensington Campus, Imperial College
London,\newline{}020 7594 5045\newline{}s.laizet@imperial.ac.uk}{}{}

\cventry{}{Dr. Andrew J. Marquis}{Senior Lecturer}{\newline{}Dept. Mechanical
Engineering,\newline{}527, City and Guilds Building, South Kensington Campus,
Imperial College London,\newline{}020 7594
7040\newline{}a.marquis@imperial.ac.uk}{}{}

% \cventry{}{Dr. Philippa. Cann}{Principal Research Fellow}{\newline{}Dept. Mechanical
% Engineering,\newline{}668, City and Guilds Building, South Kensington Campus,
% Imperial College London,\newline{}020 7594
% 7027\newline{}p.cann@imperial.ac.uk}{}{}


%----------------------------------------------------------------------------------------
%	COVER LETTER
%----------------------------------------------------------------------------------------

%\clearpage
%\recipient{Dr. Laizet, Department of Aeronautics}{Imperial College London} % Letter recipient
%\date{\today} % Letter date
%\opening{Dear Dr. Laizet,} % Opening greeting
%\closing{Yours faithfully,} % Closing phrase
%\enclosure[Attached]{curriculum vit\ae{}} % List of enclosed documents
%\extrainfo{}
%\makelettertitle % Print letter title
%
%\begin{minipage}{\textwidth}
%Please find enclosed my CV in application for the position of Research Assistant in An adjoint solver for variable-density flows in the low Mach number limit at the Department of Aeronautics.\\
%
%I am currently finishing a PhD in computational fluid dynamics (CFD) under the
%supervision of Prof. van Wachem and Dr. Marquis at the Department of Mechanical
%Engineering, Imperial College London.
%Prior to starting my PhD, I graduated from the MEng. programme in mechanical engineering at Imperial College London.
%During my undergraduate studies, I undertook courses in Computational Fluid
%Dynamics and Finite Element Analysis in addition to the further maths and fluid
%dynamics courses amongst others.
%In my final year project, I impelemented a code to solve the two-fluid model,
%supervised by Dr. Marquis, sparking an interest which directed me to pursue the
%PhD project I am currently undertaking.\\
%
%My PhD project focuses on the development of an implicit framework for the
%two-fluid model with the aim of enhanced stability for simulating complex
%gas-solid flows.
%I have implemented this in the in-house CFD code \texttt{MultiFlow}, a `coupled
%balanced-force numerical framework for single- and multi-phase flows at all
%speed on unstructured meshes', written in C and interfacing with the PETSc
%library.
%This has given me a broad experience of working with a complex, parallel code
%base and I have had the opportunity to work on a range of features in the code,
%becoming a major contributor to the code.
%
%
%
%The PhD project I have since been part of focuses on the development of
%numerical methods for interfacial two-phase flows, with the objective
%of making the accurate modelling of multi-scale interfacial flows in
%complex geometries feasible.
%	To that end, I have contributed to the development of the in-house CFD code \texttt{MultiFlow}, a `coupled balanced-force numerical framework for single- and multi-phase flows at all speeds on unstructured meshes' written in C and using the MPI library for its parallelisation. Having been involved in the making of \texttt{MultiFlow} since its early-days, I have been one its main contributors with 66,000 lines of code written, out of the 285,000 lines the code is made of (to this date). This 4-year long immersion into the development of a parallel CFD code gave me an unvaluable opportunity to touch upon the various specificities of such programmes. Alongside my involvement on the general features of \texttt{MultiFlow}, my research has focused on the developement of numerical methods for interfacial two-phase flows, and has so-far lead to one publication in a peer-reviewed journal -- with three additional papers being currently under review -- as well as eight presentations at international conferences. My involvement in the department also includes teaching assistant roles in the 2nd year mathematics and 4th year CFD courses, and the co-supervision of an MSc thesis.\\
%
%I believe that my experience with computational fluid dynamics and my awareness on environmental fluid dynamics and turbulence have prepared me for the position you are offering. I am a hard working and passionate researcher, eager to learn but also to share my knowledge. I am fully aware of the commitment that a position at your renowned institution requires, and I want to assure you that I am ready to throw myself entirely into it.\\
%%
%%Please do not hesitate to contact me should you need any additional information; I am looking forward to hearing from you.\\
%
%\end{minipage}
%
%\makeletterclosing % Print letter signature
%%----------------------------------------------------------------------------------------


\end{document}

%%% Local Variables:
%%% mode: latex
%%% TeX-master: t
%%% End:
