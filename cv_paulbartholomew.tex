\documentclass[10pt,a4paper]{moderncv} % Font sizes: 10, 11, or 12; paper sizes: a4paper, letterpaper, a5paper, legalpaper, executivepaper or landscape; font families: sans or roman

\moderncvstyle{classic} % CV theme - options include: 'casual' (default), 'classic', 'oldstyle' and 'banking'
\moderncvcolor{black} % CV color - options include: 'blue' (default), 'orange', 'green', 'red', 'purple', 'grey' and 'black'

\usepackage[utf8]{inputenc}
\usepackage{geometry} % Reduce document margins
\geometry{a4paper,tmargin=1.2cm,bmargin=1.5cm,lmargin=1.5cm,rmargin=1.5cm,headheight=2.2cm,headsep=0.5cm,footskip=0.5cm}
\setlength{\hintscolumnwidth}{1.8cm} % Uncomment to change the width of the dates column
%\setlength{\makecvtitlenamewidth}{10cm} % For the 'classic' style, uncomment to adjust the width of the space allocated to your name
\renewcommand\familydefault{cmss}

\firstname{Paul} % Your first name
\familyname{Bartholomew} % Your last name

\address{26 Viewfield Road}{London, SW18 5JE}
\mobile{+44 (0)7722926927}
\email{paul.bartholomew08@imperial.ac.uk}

%----------------------------------------------------------------------------------------

\begin{document}

\makecvtitle % Print the CV title
\vspace{-0.6cm}

%\section{Application For}
%\cvitem{}{{\fontsize{12}{12.5}\selectfont Application for eCSE funding}}
%\vspace{5mm}

I am a Post-Doctoral Research Associate (PDRA) at Imperial College London with six years' experience
working with, and developing, Computational Fluid Dynamics software for research.
In the course of this research I have developed extensive skills working with research software and
programming in widely used scientific languages including \texttt{C}, \texttt{Fortran} and
\texttt{Python} and a passion for developing scientific software.
\\

%----------------------------------------------------------------------------------------
%	EDUCATION SECTION
%----------------------------------------------------------------------------------------

\section{Education}

\cventry{2013 -- 2017}{PhD in Computational Fluid Dynamics}{Department of Mechanical Engineering,
  Imperial College London}{}{}{
  \begin{itemize}
  \item Contributed to the development of a general-purpose CFD research code, implementing a
    fully-coupled solver for the two-fluid model.
    \begin{itemize}
    \item Learned to write \texttt{C} and write software intended for parallel computing
    \end{itemize}
  \item Published in international peer-reviewed journal
  \item Presented work at national and international conferences
  \item Contributed to teaching of MSc students as a Graduate Teaching Assistant for the CFD course
  \end{itemize}
}
\cventry{2009 -- 2013}{MEng. (First class) in Mechanical Engineering}{Imperial College London}{}{}{
  \begin{itemize}
  \item Achieved the 3rd year Dean's list
  \end{itemize}
}
\cventry{2007 -- 2008}{Advanced Highers}{AAA in Chemistry, Maths and Physics}{}{}{
  \begin{itemize}
  \item Senior Proxime Accessit, Bearsden Academy
  \end{itemize}
}
\cventry{2006 -- 2007}{Highers}{AAAA in Chemistry, Maths, Physics and Technical Studies; B in
  English}{}{}{}
\cventry{2004 -- 2006}{Standard Grades}{Grade 1 in Art, Chemistry, French, Geography, Maths, Physics
  and Technical Studies; grade 2 in English}{}{}{}

\section{Professional/research experience}

\cventry{2018 -- 2019}{eCSE 13-03 \textit{A high-order accurate solver for free-surface
    flows}}{}{Imperial College London}{}{
  \begin{itemize}
  \item Awarded additional funding to continue work of eCSE 10-02 to develop a free-surface solver
  \item Key contributor to project to modernise \texttt{Incompact3d} codebase
  \item Presented work at national and international conferences
  \end{itemize}
}
\cventry{2017 -- 2018}{eCSE 10-02 \textit{An adjoint solver for variable-density flows in the low Mach
    number limit}}{}{Imperial College London}{}{
  \begin{itemize}
  \item Implemented a low Mach number solver in open-source CFD code \texttt{Incompact3d}
    \begin{itemize}
    \item Learned to write \texttt{Fortran}
    \item Gained experience working with Tier-1/0 super computers
    \end{itemize}
    \item Presented work at national conferences
    \item Published in international peer-reviewed journal
  \end{itemize}
}

\cventry{2012}{Undergraduate Research Opportunities Programme}{}{Imperial College London}{}{
  \begin{itemize}
  \item Won funding to join a research group in the Thermofluids division of the Mechanical
    Engineering department at Imperial College for the summer between the third and final year of my
    undergraduate MEng. degree.
  \item Gained experience working with Paraview
  \item Developed simple CFD code in Python
  \end{itemize}
}

\cventry{2008 -- 2009}{Year In Industry}{BAE Systems}{Glasgow}{}{
  \begin{itemize}
  \item Worked in the operations department at the Scotstoun shipyard
  \item Implemented a requisition tracking system to facilitate transfer of materials between
    projects
  \item Assisted in project management of the charity project to refit the Seagull barge
  \end{itemize}
}

\section{Publications}

\cventry{}{P. Bartholomew, S. Laizet}{A New Highly Scalable, High-Order Accurate Framework for
  Variable-Density Flows: Application to Non-Boussinesq Gravity Currents in Computer Physics
  Communications
}{2019}{}{}
\cventry{}{P. Bartholomew, F. Denner, M. H. Abdol-Azis, A. Marquis, B. van Wachem}{Unified
  Formulation of the Momentum-Weighted Interpolation for Collocated Variable Arrangements in Journal
  of Computational Physics
}{2018}{}{}

\pagebreak

\section{Skills and interests}

\cventry{}{Technical knowledge}{}{}{}{
  I have a very strong background in numerical software, particularly CFD, having worked with and
  extended two research codes each using different numerical methods.
  This experience allows me to not only contribute to the development of new software, but also to
  effectively troubleshoot problems.
  It has also guided me to identify areas where improvements can be made, for example when I decided
  to collaborate with another PDRA to produce a post-processing library from scripts we had
  developed independently: encouraging code-reuse and is already contributing to the work of others
  in the research group.\\
}
\cventry{}{Computer programming and skills}{}{}{}{
  I have successfully developed software in several of the major languages used in computational
  science including \texttt{C}, \texttt{Fortran} and \texttt{Python} and have experience using
  \texttt{MATLAB}/\texttt{Octave}.
  Much of my experience programming has been for distributed systems and I have a good knowledge of
  \texttt{MPI} and the \texttt{PETSc} library.
  I am familiar with Linux, having used it extensively throughout my research career including
  assisting with the administration of the group's workstations during my PhD and having used
  multiple HPC systems.\\
}
\cventry{}{Communication skills}{}{}{}{
  As a PDRA and PhD student I have developed strong presentation skills, presenting my work to both
  specialist and non-specialist audiences at several conferences including RSE London South East and
  EuroHPC 2019.
  In addition, working as a Graduate Teaching Assistant during my PhD required being able to explain
  concepts clearly to students during tutorial sessions.
  I have been able to combine these skills with my knowledge of software development practices to
  produce a short introduction to git to present to PhD students in the group.\\
  \linebreak
  In addition to my thesis, I have published my work in peer-reviewed journals, developing precise
  and clear writing skills in the process.
  I have also been able to employ these skills in software development and day-to-day research,
  having read about literate programming I think it has great potential for scientific software and
  have used it to produce reports with integrated post-processing code that my supervisor was very
  pleased with and am exploring its use to develop a new module for our codebase.
  Furthermore, I will be presenting results from this effort to the wider research software
  engineering community at the RSE2019 conference in Birmingham.\\
}
\cventry{}{Teamwork and collaboration}{}{}{}{
  During both my PhD and PDRA positions I have worked as part of a research group using a common
  software for research, working effectively as part of a team has thus been vital.
  With multiple people collaborating on a single codebase I have also become proficient in both svn
  and git, finding these skills valuable also when working independently.\\
}
\cventry{}{Personal interests}{}{}{}{
  I play squash and badminton and cycle both to commute and recreationally.
  I also enjoy playing guitar and have played in several bands.
}

% \section{Conference and seminar presentations}

% \cventry{May 2019}{P. Bartholomew, G. Deskos, S. Laizet}{Xcompact3d: A Powerful Framework to Study
%   Turbulent Flows with Turbulence-Resolving Simulations}{EuroHPC, 13-17 May 2019}{Pozna\'{n},
%   Poland}{}
% \cventry{Apr 2019}{P. Bartholomew, S. Laizet}{Modernising and expanding the capabilities of the
%   high-order flow solver Incompact3d (Poster)}{Numerical Algorithms for High-Performance
%   Computational Science, 8-9 February 2019}{London, United-Kingdom}{}
% \cventry{Feb 2019}{P. Bartholomew, S. Laizet}{Modernising and expanding the capabilities of the
%   high-order flow solver Incompact3d (Poster)}{RSLondonSouthEast Workshop 2019, 7 February
%   2019}{London, United-Kingdom}{}
% \cventry{Sep 2018}{P. Bartholomew, S. Laizet}{QuasIncompact3D: A Highly Scaleable Solver for Navier
%   Stokes in the Low Mach Number Limit}{UK Fluids Conference, 4-6 September 2018}{Manchester,
%   United-Kingdom}{}
% \cventry{Apr 2018}{P. Bartholomew, S. Laizet}{Simulations of Variable-Density Flows in the Low Mach
%   Number Limit}{Incompact3D User Group Meeting 2018, April
%   2018}{London, United-Kingdom}{}
% \cventry{Sep 2016}{P. Bartholomew, A. Marquis, B. van Wachem}{Modelling of
%   Turbulent Gas-Solid Flows in the Eulerian Framework}{Inaugural UK Fluids
%   Conference 2016, 7-9 September 2016}{London, United-Kingdom}{}

% \cventry{May 2016}{P. Bartholomew, A. Marquis, F. Denner, B. van Wachem}{Development of a Fully
%   Coupled Solver for the Eulerian-Eulerian Approach}{9th International Conference on Multiphase Flow
%   (ICMF 2016), 22-27 May 2016}{Florence, Italy}{}

% \section{Teaching activities}

% \cventry{2015-2017}{CDT - Computational Fluid Dynamics}{Teaching assistant}{Imperial College London}{}{}

% \vspace{0.3cm}
% \begin{minipage}[t]{0.45\textwidth}
% \section{Computer skills}
% \cvitemwithcomment{Languages}{C, Fortran, Python, \LaTeX, Matlab}{}
% \cvitemwithcomment{Software}{Linux, MPI, Paraview, PETSc}{}
% \end{minipage}
% \hspace{0.7cm}
% \begin{minipage}[t]{0.45\textwidth}
% \setlength{\hintscolumnwidth}{3.4cm} 
% \setlength{\hintscolumnwidth}{3cm} 
% \end{minipage}

% %----------------------------------------------------------------------------------------
% %	INTERESTS SECTION
% % ----------------------------------------------------------------------------------------

% \section{Interests}

% \renewcommand{\listitemsymbol}{$\bullet$~} % Changes the symbol used for lists

% % \cvitemwithcomment{Sports}{Running, Coaching, Rowing}{}
% \cvitemwithcomment{Sports}{Cycling, Badminton, Squash}{}
% \cvitemwithcomment{Hobbies}{Guitar}{}

\section{Referees}

\cventry{}{Dr. Sylvain Laizet}{Senior Lecturer}{\newline{}Dept. Aeronautical
  Engineering,\newline{}339, City and Guilds Building, South Kensington Campus, Imperial College
  London,\newline{}020 7594 5045\newline{}s.laizet@imperial.ac.uk}{}{}

\cventry{}{Dr. Andrew J. Marquis}{Senior Lecturer}{\newline{}Dept. Mechanical
  Engineering,\newline{}527, City and Guilds Building, South Kensington Campus, Imperial College
  London,\newline{}020 7594 7040\newline{}a.marquis@imperial.ac.uk}{}{}

% \cventry{}{Dr. Philippa. Cann}{Principal Research Fellow}{\newline{}Dept. Mechanical
% Engineering,\newline{}668, City and Guilds Building, South Kensington Campus,
% Imperial College London,\newline{}020 7594
% 7027\newline{}p.cann@imperial.ac.uk}{}{}

\end{document}

%%% Local Variables:
%%% mode: latex
%%% TeX-master: t
%%% End:
